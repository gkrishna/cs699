% Chapter Template

\chapter{Conclusion and Future Work} % Main chapter title

\label{conclusion} % Change X to a consecutive number; for referencing this chapter elsewhere, use \ref{ChapterX}

\lhead{Chapter 6 . \emph{Conclusion and Future Work}} 

In this work, we have demonstrated that if we use social 
networking meta data related to images present on social networks 
like Flickr, Instagram, Facebook, Google Plus etc. We achieve much 
better image classification results. We started with five 
different visual features like SIFT, GIST, HoG, Color Histogram and 
GLCM. First, we classified our data set based on these individual 
features and then we took an ensemble model developed by a weighted 
combination of these feature models. This analysis shows that for image 
classification problems individual features are less effective. We 
need to combine these features for better classification.

We have further presented a simple and efficient approach of 
converting the social meta data of images to feature vectors. We used 
text processing methods to extract the semantic structure of 
these social feature vectors in low dimensional spaces. These  social 
feature vectors gave better classification for most labels 
compared to visual features. This phenomena can be explained by the 
fact that the social feature vector is actually constructed from 
social meta-data, which is very similar to textual information 
about the image. The comments, the tags, the groups and the 
galleries of an image has direct textual information of that 
image class. Therefore, correlating this textual information with 
the class of the image is not so complex. Where as in case of visual 
features, we have to do a lot of processing to get features, 
which embody some visual property like texture, objects, color 
pattern and spatial envelope. But none of these properties directly 
tell the exact label of the image. In contrast in social meta data we 
might have the label as a tag or as a word in a comment.

Several machine learning techniques were explored for analysis and 
classification including Spatial Pyramid, Bag of Words Model, Latent 
Semantic Indexing and LibSVM. The results from these techniques came 
out to be much better then the best results of the competitions held 
on these four datasets. We, therefore, can conclude that social-
meta data gives that extra information, which proves to be  
instrumental in image classification. The fusion of social features 
and visual features enhances the accuracy of image classification in
all cases that we examined and sometimes the difference is very 
significant.
\section{Future Work}
Some of the labels in our data-set did not have enough images like 
fox, lake, mountain, protest, earthquake etc. In such cases, when the 
number of images available were less then 50, we discarded that 
label for classification because we did not have enough positive 
instances. In such cases we can try web-information retrieval 
methods to get more images for such labels.

We used the strategy of extracting the features, using 
dimensionality reduction techniques like Spatial Pyramid, Latent 
Semantic Indexing, Bag of Words etc  and then used classifiers 
like LibSVM on these low dimensional features.
This problem can also be explored using Deep Belief Networks 
\citep{dbn} with different number of hidden states, stacks of 
Restricted Boltzmann Machines and layers to get prominent reduced 
features which can further be classified using supervised 
classification techniques.

We have currently used MATLAB and Python in our implementation.
This implementation was done on single thread on a local system. We 
can change our implementation to faster platforms C++, 
with parallel processing for computationally difficult parts of 
feature extraction. This can considerably speed up model creation. 
We used the approach of weighted features, we 
could use better fusion techniques to ensemble all the features.

Our social feature vector will be efficient only if the social meta 
data of an image already has enriched textual information in it. To 
overcome this constraint we can explore the dimension of using the 
network structure of social meta data and co-relating images based
on the network. This technique can also be used to 
retrieve images from social networking sites on the basis of 
linkages. The system can be extended to do automated tag recommendations. 
