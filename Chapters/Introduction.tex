% Chapter Template
\chapter{Introduction} % Main chapter title
\label{INTR} % Change X to a consecutive number; for referencing this chapter elsewhere, use \ref{ChapterX}
\lhead{Chapter 1. \emph{Introduction}}
It has become a cliche to start a discussion on the field related to
classification or tagging of online multimedia information by
stating just how the meteoric growth of data has been in recent
decades. While this may be a very mundane way to introduce the need
for annotated online digital corpora and the quantum of effort
required, it is nonetheless, true for work in this field.

The proliferation of personalized gadgets, cheap digital cameras and
diversification away from single use devices like old generation mobile
phones to new generation smart phones, tablets and wearable devices has
spawned a culture of documenting and saving every moment of our lives in
digital form. Internet services like Facebook, Flickr,
Instagram etc also make it very easy for anyone to share
their pictures with their (online) social connections.

In the present scenario, where we have several robust systems
effectively handling billions of photos, images and videos; the
renaissance of the `socialization' of web activity has produced a
massive amount of social interaction information. The social
interaction on multimedia supporting websites have also opened a new
avenue of managing this ever growing corpora of multimedia data. The
point is that this social interaction data like tags, comments, likes,
groups, galleries, playlists etc. all leave some clues about the
content in question.

This social meta-data, the data related to the content, can be a
pathway to organize the whole multimedia corpus through a labelling
exercise using classification, theming and ontology creation and use.
This meta-data can help us in many ways for example in
information retrieval, multimedia classification, heterogeneous
learning etc. In this thesis, we focus on images and their social meta data.

Photos unlike text documents have a far more immediate
emotive impact - reportage from people dying in Somalia because of
poverty to striking wild life photography can make an impression on
viewers very quickly. In the past, photographs were normally confined to
an album and were rarely opened to relive old memories. The
photos were difficult to share and normally did not contain any
contextual information. Whereas today people can
upload images to the world wide web directly from capture devices
(like phones, tablets). They can also encode pertinent contextual
information automatically and share it with all their social network
contacts. This starts a cycle of interaction and annotation of that
image. Over time, the social aspect of the image becomes more
prevalent. This creates a need for expanding our understanding and
power of utilizing this social data.

This thesis focuses on exploring the role which can be played
by social context information in enhancing image classification.
In image classification, an image is classified according to its
visual content. For example does it contain clouds or not. We
analyze some of the visual characteristics based features and then,
we try to fuse these visual features with social context
information to follow a multimodal approach to classification.
\section{Motivation}
Image classification has many application ranging from multimedia
information delivery to web search. In the past, image
classification has faced two major difficulties. First, the labeled
images for training are generally hard to extract and in short
supply also labeling new images is costly and requires significant
human input. Second, images can be ambiguous; e.g. an image can have
multiple concepts hidden.

To overcome these problems, we can leverage the other information
about the images. Even when labeled data is hard to create, we can
take advantage of the text data related to images or the social
interaction data associated with images.

Apart from these problems, we also experience the problem of
ambiguity of key objects in image classification. The low level
features of visual information can often be ambiguous. An object can
belong to more than one concept (polysemy). In such cases, we need
additional data to gracefully handle such ambiguous
conditions in order to get high classification accuracy.

Text and images are two distinct types of information resource and
belong to two different modalities as they present an object in
different ways. However, there are some implicit connections and
invariant properties shared between images and textual information.
In fact, the texts associated with images include some form of human
generated description of images; thus these can be regarded as a
supplement to the image content.

The combination of textual information with image features can be a
way of improving image classification results. With all these cues
about heterogeneous learning, we wish to take the help of data in
different modalities to have better classification of images. The
unprecedented evolution of social networking gives us such data with
reasonable ease. People talk about images, put likes, comments,
tags and other markers. These markers or rather social meta-data can
be instrumental in inferring the content of images.

 In the light of all these motivating facts, we decided to pursue
social meta data as an additional source of information for
enhancing image classification.

\section{Problem Statement}
In this thesis, we will focus on classification of images, which
have some social meta-data available with them. The social meta data
for an instance can be  tags on images, groups in which image is
featured or comments on the image etc.

If we try to break an image classification problem for multi-labeled
images, we can talk in terms of labeling an image with different
labels. We will be using supervised learning techniques to do this,
which means we will first develop a classification system by
creating classification models using ground truth labeled images and then
use these models to automatically label the unknown images with the
label.

Annotating an image  with a label can be considered as a multi-
class classification problem where the classes are represented by
labels. For example an image with a ship floating in the sea, can be labeled
as "sea", "ship" or "water". So, the problem of annotating an image
with its labels is equivalent to classifying that image into one of
the labels. We can define a problem as follows:

	Suppose we have an image data set consisting of N images $X = \{x_1 ...x_N\}$ and a label space consisting of L categories $L = \{-1, 1\}^L$.
	
	We can denote the ground truth labeling for the image $x_n$ as $y^n\in L$. Then the ground truth for a particular category $c$ can be denoted as $y^n:y^n \in \{-1,1\}$. When we combine the ground truth for the entire data set for category c, we get $Y_c \in \{-1,1\}^N$.
	
We try to learn a prediction made for an image $x_n$ and category $c$,
which will be $\bar{y_c} (x_n,\theta_c)\in\{-1,1\} $. Predictions
across the entire data set for category $c$ will be
$\bar{Y_c}(X,\theta_c) \in \{-1,1\}^N$.
	
	 We propose methods to enhance image classification using
social meta-data. If we only have images and no auxiliary data ,
we will only get the visual features  $\theta_c ^ {Visual} \in R^{Visual\ features}$ for an image. We, therefore, first consider some widely used visual features (SIFT Features, GIST Features,  COLOR Space Features, Texture/GLCM Features and HOG-LBP Features) to learn a classification model on these images.
	
	   We, then, test our visual-only classification model, to see how
accurate it is in classifying the images. Now, we consider the
features generated by the social features
$\theta_c ^{Social} \in R^{Social\ features}$ for an image to  learn
a classification model on these images.  We, then, compare our
results of social-only classification model to visual-only
classification model.
	
	   In the third step, we ensemble these visual-only and social-only
classifiers to obtain the final result. We compare all the results
to find the improvement obtained by using some auxiliary social-
meta data.

\section{Outline of the Thesis}
The remaining thesis is structured as follows:
\begin{itemize}
\item{{\bf Chapter 2} reviews the related work done in utilizing the meta-data and other information related to images in order to enhance image classification.}
\item{{\bf Chapter 3} gives an overview of the data-sets, we considered in our thesis. }
\item{{\bf Chapter 4} describes the feature extraction, which is quite important in our experiments.}
\item{{\bf Chapter 5} presents the experiments we performed and their results. }
\item{{\bf Chapter 6} contains conclusions and pointers to extensions and future work. }
\end{itemize}

%{\bf *** Add a precise statement of the problem and a separate section at
%the end on the organization of the rest of the thesis. Since you have a
%separate chapter on related work remove the mentions of related work here
%and put all of it in the chapter on related work.
%You should use proper cite tags for references and use bibtex with a
%suitable style file and not give absolute references as you have done
%for some citations. Do not use \\ for para separation, instead use
%a blank line. ***}
